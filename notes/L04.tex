\documentclass[11pt]{article}
\usepackage{textcomp}
\usepackage{listings}
\usepackage{tikz}
\usepackage{enumerate}
\usepackage{url}
%\usepackage{algorithm2e}
\usetikzlibrary{arrows,automata,shapes}
\tikzstyle{block} = [rectangle, draw, fill=blue!20, 
    text width=5em, text centered, rounded corners, minimum height=2em]
\tikzstyle{bt} = [rectangle, draw, fill=blue!20, 
    text width=1em, text centered, rounded corners, minimum height=2em]

\lstset{ %
language=Java,
basicstyle=\ttfamily,commentstyle=\scriptsize\itshape,showstringspaces=false,breaklines=true,numbers=left}

\newtheorem{defn}{Definition}
\newtheorem{crit}{Criterion}

\newcommand{\handout}[5]{
  \noindent
  \begin{center}
  \framebox{
    \vbox{
      \hbox to 5.78in { {\bf Intro to Methods of Software Engineering } \hfill #2 }
      \vspace{4mm}
      \hbox to 5.78in { {\Large \hfill #5  \hfill} }
      \vspace{2mm}
      \hbox to 5.78in { {\em #3 \hfill #4} }
    }
  }
  \end{center}
  \vspace*{4mm}
}

\newcommand{\lecture}[4]{\handout{#1}{#2}{#3}{#4}{Lecture #1}}
\topmargin 0pt
\advance \topmargin by -\headheight
\advance \topmargin by -\headsep
\textheight 8.9in
\oddsidemargin 0pt
\evensidemargin \oddsidemargin
\marginparwidth 0.5in
\textwidth 6.5in

\parindent 0in
\parskip 1.5ex
%\renewcommand{\baselinestretch}{1.25}

\newcommand{\squishlist}{
 \begin{list}{$\bullet$}
  { \setlength{\itemsep}{0pt}
     \setlength{\parsep}{3pt}
     \setlength{\topsep}{3pt}
     \setlength{\partopsep}{0pt}
     \setlength{\leftmargin}{1.5em}
     \setlength{\labelwidth}{1em}
     \setlength{\labelsep}{0.5em} } }
\newcommand{\squishlisttwo}{
 \begin{list}{$\bullet$}
  { \setlength{\itemsep}{0pt}
     \setlength{\parsep}{0pt}
    \setlength{\topsep}{0pt}
    \setlength{\partopsep}{0pt}
    \setlength{\leftmargin}{2em}
    \setlength{\labelwidth}{1.5em}
    \setlength{\labelsep}{0.5em} } }
\newcommand{\squishend}{
  \end{list}  }

\begin{document}

\lecture{4 --- October 4, 2016}{Fall 2016}{Patrick Lam}{version 1}

\section*{Written Communication}

\noindent
``What\textinterrobang Engineers need to communicate?''

\noindent
Good communication is always vital to success---
\squishlist
\item it's not about being able to calculate derivatives;
\item it's not even about being able to write code.
\squishend

\noindent
The following skills are also important, but we won't talk about them today.
\squishlist
\item active and emphathetic listening; and,
\item public (and less-public) speaking [SPCOM 100/223].
\squishend

\noindent
How to get better at writing (and any other skill):

\begin{center}
{\Large \bf practice!} \\
(and get feedback and read others' writing)
\end{center}

\vspace*{1em}

\noindent
``Writing skill'' is a misnomer. Your skill depends on your familiarity with the genre.

\noindent
Important genres for SE:
\squishlist
\item code comments;
\item commit messages / pull requests / bug reports; and
\item emails!
\squishend

\noindent
Not so important: English essays, technical reports.

\section*{Good technical writing}
Jean-luc Doumont proposes that good technical writing should be:
\squishlist
\item clear;
\item accurate; and
\item concise.
\squishend
\pagebreak
Similarly, Derek Rayside proposes:
\squishlist
\item clear;
\item concise;
\item complete; and
\item correct.
\squishend

Good writing is reader-friendly: the writing does not get in your way and you can focus
on the ideas in the text.

Before writing, consider the following questions:
\squishlist
\item who is the audience?
\item what action are you hoping for?
\squishend

\paragraph{Case study.} Pull requests consist of a short message associated with some code that should be merged into the main line of a code repository.
For a pull request, the audience is your peers, and you are hoping that they will merge your code.

Here is an example pull request message from a 2014 Capstone Design Project\footnote{\url{https://github.com/TeamAmalgam/kodkod/pull/37}}. 

\begin{quote}
Currently if an uncaught exception occurs in a multiobjective algorithm the thread will be terminated but the main thread will continue waiting for solutions indefinitely.

This change allows us to add exceptions to the blocking queue which will be rethrown by the BlockingSolutionIterator, thus terminating the main thread when the solver thread dies.

This change will require modifications to all MultiObjectiveAlgorithms as it changes the interface.
multiObjectiveSolve is no longer abstract, instead the abstract template method multiObjectiveSolveImpl should be implemented by the various algorithms.

Reviewers: @AtulanZaman @joseph2625 @mhyee
\end{quote}
Note: 1) the use of paragraphs, each with a message; 2) descriptions of: the current state; the effects of the change; and work that will be required as a result of the change; and, 3) code reviews.

\paragraph{Clarity in technical writing.} Use the right words and put them in the right order. ``Use definite, specific, concrete language.'' (\emph{Elements of Style \#16}; example: ``A period of unfavorable weather set in.'' versus ``It rained every day for a week.'')

\paragraph{Accuracy.} Get the details right. Say who is doing the action (and avoid passive voice: ``It was decided to replace the professor with a robot.'')

\paragraph{Concision.} As \emph{Elements of Style \#17} puts it: ``Omit needless words'', for example ``Her story is a strange one.'' versus ``Her story is strange.''

\end{document}
